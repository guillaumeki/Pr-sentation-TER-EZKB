

\begin{frame}{Chaînage avant (\textit{chase})}
    Le chaînage avant consiste à inférer des nouveaux faits à partir de la base de faits et de la base de règles, jusqu'à qu'il ne soit plus possible d'en inférer.
    \begin{block}{Exemple}
        $\mathcal{KB} = (\mathcal{F}, \mathcal{R})$
        \\ $\mathcal{F} = \{estHumain(socrate)\}$
        \\ $\mathcal{R} = \{r_1 = estHumain(X) \rightarrow estMortel(X)\}$.
        \\ On peut inférer : $estMortel(socrate)$.
        \\ On obtient $\mathcal{F}^* = \{estHumain(socrate), estMortel(socrate)\}$, et on ne peut plus rien inférer de nouveau.
    \end{block}
\end{frame}

\begin{frame}{Déclencheur, application}
    \begin{block}{Déclencheur}
        Un délencheur est un couple $(R,h)$ où $R = B \rightarrow H$ est une règle et $h$ un homomorphisme permettant d'envoyer $B$, le corps de la règle $R$, sur la base de faits.
    \end{block}
    \begin{block}{Application d'un déclencheur}
        Pour appliquer $(R,h)$, on crée une substitution $h^{safe}$ qui est une extension de $h$ où les variables existentielles de la tête de la règle sont renommées avec une nouvelle variable fraîche.
    \end{block}
    \begin{block}{Exemple}
        $\mathcal{KB} = (\mathcal{F} = \{p(a,b)\}, \{r_1 = p(X,Y) \rightarrow p(X,Z)\})$ \\
        $(r_1,h = \{(X \mapsto a), (Y \mapsto b)\})$ et
        $h^{safe} = \{(X \mapsto a), (Y \mapsto b), (Z \mapsto Z_0)\}$ \\
        $\mathcal{F}^* = \mathcal{F} \cup h^{safe}(p(X,Z)) = \{p(a,b),p(a,Z_0)\}$
    \end{block}
\end{frame}

\begin{frame}{Chaînage avant en largeur}
    \begin{itemize}
        \item On cherche tous les déclencheurs sur une base de faits $\mathcal{F}_i$ ;
        \item On les applique tous pour obtenir une base $\mathcal{F}_{i+1}$ ;
        \item On recommence tant qu'il est possible d'ajouter des nouveaux faits.
    \end{itemize}
\end{frame}