\begin{frame}{Core chase}
\begin{itemize}
    \item Le critère d'applicabilité est le même que celui du \textit{restricted chase} ;
    \item On rajoute en plus le calcul du \textit{core} de la base de faits à la fin de chaque étape.
\end{itemize}
    
\end{frame}

\begin{frame}{Exemple d'exécution du \textit{core chase}}
    
$\mathcal{F}=\{ p(a,b)\}$ \\
$\mathcal{R} = \{ R = p(X,Y) \rightarrow q(Y,Y),q(Y,Z) \}$

\begin{center}
\begin{tikzpicture}[->,>=stealth',shorten >=1pt,auto,node distance=2.8cm,
                    semithick]
  \tikzstyle{every state}=[fill=green!20,circle,text=black,minimum size=0.7cm]

  \node[state]				(A)              			{$a$};
  \node[state]				(B)  [right of=A]  			{$b$};
  \node[state,fill=red]		(C)  [right of=B]			{$Z_1$};
    
  \path (A) edge 								node[above] {$p$} 	(B);
  \path (B) edge 	[loop]						node[above] {$q$} 	(B);
  \path (B) edge 	[color=red]					node[above] {$q$} 	(C); 
  
  
\end{tikzpicture}
\end{center}


\end{frame}