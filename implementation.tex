\begin{frame}{Objectifs de la nouvelle implémentation}
    \begin{itemize}
        \item Proposer une implémentation destinée à remplacer celle existante.
        \item Être suffisamment flexible pour pouvoir implémenter tous les chaînages avant étudiés et pour pouvoir évoluer facilement ;
        \item Être mieux structurée que l'implémentation existante ;
        \item Essayer d'améliorer les performances ;
    \end{itemize}
    Ces objectifs ont été globalement atteints.
\end{frame}

\begin{frame}{Structuration}
    \begin{itemize}
        \item Des classes principales implémentant le chaînage avant ;
        \item Auxquelles on fournit :
        \begin{itemize}
            \item Une classe applicatrice des règles (flexibilité importante, par exemple version spécifique à une base de donnée) ;
            \item Des classes implémentant les extensions (locale et globale) ;
            \item Une classe vérifiant le critère d'applicabilité.
        \end{itemize}
    \end{itemize}
\end{frame}