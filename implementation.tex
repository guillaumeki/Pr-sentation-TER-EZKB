\begin{frame}{Objectifs de la nouvelle implémentation}
    \begin{itemize}
        \item Proposer une implémentation destinée à remplacer celle existante.
        \item Être suffisamment extensible pour pouvoir implémenter tous les chaînages avant étudiés et pour pouvoir évoluer facilement ;
        \item Être mieux structurée que l'implémentation existante ;
        \item Essayer d'améliorer les performances ;
    \end{itemize}
    Ces objectifs ont été globalement atteints.
\end{frame}

\begin{frame}{Structuration et implémentation}

\begin{algorithm}[H]\label{algo:chainage_avant_largeur}
\setstretch{1}
\caption{Chaînage avant en largeur}
\SetAlgoLined
\DontPrintSemicolon
{\footnotesize
\SetKwInOut{Input}{entrées}
\SetKwInOut{Output}{sortie}
\Input{une base de faits $\mathcal{F}$, une base de règles $\mathcal{R}$}
\Output{la base de faits saturée $\mathcal{F^*}$}
\SetKwRepeat{Do}{faire}{tant que}

$\mathcal{F}^* \gets \mathcal{F}$\;

\Do{$nouvFaits \not= \emptyset$}
{
    $nouvFaits\gets \emptyset$ \;
    \ForEach{règle : $ R = B \rightarrow  H \in \mathcal{R}$} 
    {
        \ForEach{\textcolor{green}{déclencheur} $(R,h)$} 
        {
            \uIf{$\textcolor{blue}{estApplicable(R,h)}$}
            {
                $nouvFaits \leftarrow \textcolor{red}{\mbox{\textit{étendreLocalement}}(nouvFaits, (R,h))}$\;
            }
        }
    }
    $\mathcal{F}^*\gets \textcolor{red}{\mbox{\textit{étendreGlobalement}}(\mathcal{F}^*, nouvFaits)}$\;
}
\Return $\mathcal{F}^*$
}
\end{algorithm}
     
\end{frame}