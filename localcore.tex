\begin{frame}{Local core chase}
\begin{itemize}
    \item Similaire au \textit{core chase} ;
    \item la différence se situe lors du calcul du \textit{core} : on "gèle" la base de faits, on ajoute les nouveaux faits, on calcule le \textit{core} et on dégèle.
\end{itemize}
    
\end{frame}

\begin{frame}{Exemple d'exécution du \textit{local core chase}}
    
$\mathcal{F}=\{ p(X,a)\}$ \\
$\mathcal{R} = \{ R = p(X,Y) \rightarrow p(b,Y),p(b,Z) \}$

\begin{center}
\begin{tikzpicture}[->,>=stealth',shorten >=1pt,auto,node distance=2.8cm,
                    semithick]
  \tikzstyle{every state}=[fill=green!20,circle,text=black,minimum size=0.7cm]

  \node[state]				(A)              			{$a$};
  \node[state]				(B)  [below right of=A]		{$b$};
  \node[state,fill=orange]		(X)  [below left of=A]  {$X$};
  \node[state,fill=red]		(Z)  [above right of=B]		{$Z_1$};
    
  \path (B) edge 								node[above] {$p$} 	(A);
  \path (B) edge 	[color=red]					node[above] {$p$} 	(Z); 
  \path (X) edge 	[color=orange]				node[above] {$p$} 	(A); 
  
  
\end{tikzpicture}
\end{center}
\
\end{frame}