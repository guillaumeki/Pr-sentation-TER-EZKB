% Packages
\usepackage[T1]{fontenc}
\usepackage[francais]{babel}
\usepackage{array}
\usepackage{verbatim}
\usepackage{graphics} %inclusion de figures
\usepackage{graphicx} %inclusion de figures
\usepackage{pstricks,pst-node} %Graphiques
\usepackage{booktabs}
\usepackage{minted}
\usepackage{rotating}
\usepackage{wrapfig}
\usepackage{graphicx}
\usepackage[inline]{asymptote} 
\usepackage{amsmath}
\usepackage{amsfonts}
\usepackage{amssymb}
\usepackage{amsthm}
\newcommand{\eod}{\ensuremath{\hfill\dashv}}
\newcommand{\eop}{\ensuremath{\hfill\clubsuit}}
\newcommand{\fun}[1]{\ensuremath{\mbox{\it #1}}}
\newcommand{\expo}[2]{{#1}^{\mbox{\scriptsize \sl #2}}}
\newcommand{\var}{\ensuremath{\mathbf{var}}\xspace}
\newcommand{\exv}{\ensuremath{\mathbf{exv}}\xspace}
\newcommand{\dsv}{\ensuremath{\mathbf{dsv}}\xspace}
\newcommand{\term}{\ensuremath{\mathbf{term}}\xspace}
\newcommand{\pred}{\ensuremath{\mathbf{pred}}\xspace}
\newcommand{\fr}{\ensuremath{\mathbf{fr}}\xspace}
\newcommand{\cnst}{\ensuremath{\mathbf{cnst}}\xspace}
\newcommand{\Vars}{\ensuremath{\mathsf{Vars}}\xspace}
\newcommand{\Pvars}{\ensuremath{\mathsf{Pvars}}\xspace}
\newcommand{\Fvars}{\ensuremath{\mathsf{Fvars}}\xspace}
\newcommand{\Terms}{\ensuremath{\mathsf{Terms}}\xspace}
\newcommand{\Cons}{\ensuremath{\mathsf{Const}}\xspace}
\newcommand{\Preds}{\ensuremath{\mathsf{Preds}}\xspace}
\newcommand{\ic}{\ensuremath{\cdot\{}\xspace}
\newcommand{\ci}{\ensuremath{\}\cdot}\xspace}
\newcommand{\id}{\ensuremath{:\hspace{-1.56mm}\{}\xspace}
\newcommand{\di}{\ensuremath{\}\hspace{-1.56mm}:}\xspace}
\newcommand{\bet}{\ensuremath{\beta^{\diamond}}\xspace}
\newcommand{\sep}{\ensuremath{\mathbf{sep}}\xspace}
\newcommand{\vect}[1]{\mathbf{#1}}
\newcommand{\vars}[1]{\fun{vars}{(#1)}}
\newcommand{\terms}[1]{\fun{terms}{(#1)}}
\newcommand{\atoms}[1]{\fun{atoms}{(#1)}}
\newcommand{\type}[1]{\fun{type}{(#1)}}
\newcommand{\const}{a}
\newcommand{\atom}{\alpha}
\usepackage{tikz}
\usetikzlibrary{arrows,positioning,automata,shadows,fit,shapes,calc, shapes, backgrounds}

\newcolumntype{C}[1]{>{\centering\arraybackslash}p{#1cm}}

\newcommand{\pt}[5]{\dotnode(#2,#3){#1} \uput[#4](#2,#3){#5}}
\newcommand{\ptt}[5]{\dotnode(#2,#3){#1} \uput[#4](#2,#3){{\tiny #5}}}

% Style
\usepackage{pgfpages}
\usetheme{Montpellier}
\usecolortheme{whale}
\setbeamertemplate{footline}[page number]

% Informations
\title{Suppression des redondances dans les bases de connaissances}
\author{
  Leonardo \bsc{Moros} \\
  Guillaume \bsc{Perution-Kihli} \\
  Romain \bsc{Ricalens} \\
  Julien \bsc{Rodriguez} \\
  Rami \bsc{Younes}}

% Tableaux
\definecolor{GT1}{gray}{0.99}
\definecolor{GT2}{gray}{0.95}
\newcommand{\mc}[3]{\multicolumn{#1}{#2}{#3}}
\newcommand{\mr}[2]{\begin{tabular}[x]{@{}c@{}}#1\\#2\end{tabular}}
\newcolumntype{R}[1]{>{\raggedleft\arraybackslash}m{#1}}
\newcolumntype{L}[1]{>{\raggedright\arraybackslash}m{#1}}
%\newcolumntype{C}[1]{>{\centering\arraybackslash}m{#1}}

% Divers
\newcommand{\true}{\textcolor{orange}{1}}
\newcommand{\false}{\textcolor{orange}{0}}
\addto\captionsfrench{\def\figurename{{\bsc{Figure}}}}
\addto\captionsfrench{\def\tablename{{\bsc{Tableau}}}}