 \begin{frame}{Redondances inter-règles}
    Si une règle $R$ est redondante à une base de règles $\mathcal{R}$ alors elle peut être inférée par $\mathcal{R} \backslash R$.
    
    \begin{block}{Exemple}
        Soit la base de règles $\mathcal{R} = \{ R_1 = p(X) \xrightarrow{} q(X), R_2 = p(X) \xrightarrow{} \exists Z.t(X,Z), R_3 = q(X) \xrightarrow{} \exists T.t(X,T)\}$
    \end{block}
    
    Nous verrons lors de l'application d'un algorithme sur cet exemple que $R_2$ peut être inférée par $\mathcal{R} \backslash R_2$.
    
\end{frame}

\begin{frame}{Redondances inter-règles}
   
   Algorithme pour supprimer les redondances inter-règles :
    
    \begin{itemize}
        \item On prend le corps de $R$ comme base de faits $\mathcal{F}$ initiale ;
        \item On sature $\mathcal{F}$ et on obtient $\mathcal{F}^*$ la base de faits saturée ;
        \item On gèle la frontière de $R$ et on cherche un homomorphisme de la tête de $R$ dans $\mathcal{F}^*$.
    \end{itemize}
    
    La saturation d'une base de faits pouvant être infinie, il est préférable de prévoir un arrêt au bout d'un certains nombre d'étapes.
    
\end{frame}

\begin{frame}{Redondances inter-règles}
   
   Algorithme pour supprimer les redondances inter-règles :
    
    \begin{itemize}
       \color{red}
        \item On prend le corps de $R$ comme base de faits $\mathcal{F}$ initiale ;
       \color{black}
        \item On sature $\mathcal{F}$ et on obtient $\mathcal{F}^*$ la base de faits saturée ;
        \item On gèle la frontière de $R$ et on cherche un homomorphisme de la tête de $R$ dans $\mathcal{F}^*$.
        
    \end{itemize}
    
    \begin{block}{Exemple}
          Soit la base de règles $\mathcal{R} = \{ R_1 = p(X) \xrightarrow{} q(X), R_2 = p(X) \xrightarrow{} \exists Z.t(X,Z), R_3 = q(X) \xrightarrow{} \exists T.t(X,T)\}$. On traite $R_2$ et on construit la base de connaissances $\mathcal{KB} = \{ \{p(X)\}, \mathcal{R} \backslash R_2\}$.
    \end{block}
    
\end{frame}

\begin{frame}{Redondances inter-règles}
   
  Algorithme pour supprimer les redondances inter-règles :
    
    \begin{itemize}
       \item On prend le corps de $R$ comme base de faits $\mathcal{F}$ initiale ;

        \item On sature $\mathcal{F}$ et on obtient $\mathcal{F}^*$ la base de faits saturée ;
        
               
        \color{red}
        \item On gèle la frontière de $R$ et on cherche un homomorphisme de la tête de $R$ dans
        $\mathcal{F}^*$.
        \color{black}  
    \end{itemize}
    
    \begin{block}{Exemple}
          Soit la base de règles $\mathcal{R} \backslash R_2 = \{ R_1 = p(X) \xrightarrow{} q(X), R_3 = q(X) \xrightarrow{} \exists T.t(X,T)\}$. On sature $\mathcal{F} = \{p(X)\}$ et on obtient $\mathcal{F}^* = \{p(X), q(X), t(X,T)\}$. Ici $t(X,Z)$ appartient à $\mathcal{F}^*$ donc $R_2$ est redondante. On peut donc la supprimer de la base de règles.
    \end{block}
    
\end{frame}


\begin{frame}{Redondances inter-règles}
   
  Algorithme pour supprimer les redondances inter-règles :
    
    \begin{itemize}
          \item On prend le corps de $R$ comme base de faits $\mathcal{F}$ initiale ;

        \item On sature $\mathcal{F}$ et on obtient $\mathcal{F}^*$ la base de faits saturée ;
        
  
        \item On gèle la frontière de $R$ et on cherche un homomorphisme de la tête de $R$ dans $\mathcal{F}^*$.
    
   
    \end{itemize}
    
    \begin{block}{Exemple}
    $R_1$ et $R_3$ ne sont pas redondante. 
        Soit la nouvelle base de règles $\mathcal{R}' = \{ R_1 = p(X) \xrightarrow{} q(X), R_3 = q(X) \xrightarrow{} \exists T.t(X,T)\}$ sans redondance inter-règles.
    \end{block}
    
\end{frame}

