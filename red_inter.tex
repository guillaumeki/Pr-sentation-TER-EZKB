\begin{frame}{Redondances inter-règles}
    Si une règle $R$ est redondante à une base de règles $\mathcal{R}$ alors elle peut être inférée par $\mathcal{R} \backslash R$.
    
    \begin{block}{Exemple}
        Soit la base de règles $\mathcal{R} = \{ R_1 = p(X) \xrightarrow{} q(X), R_2 = p(X) \xrightarrow{} t(X), R_3 = q(x) \xrightarrow{} t(X)\}$
    \end{block}
    
    Nous verrons lors de l'application de l'algorithme sur cet exemple que $R_2$ peut être inférée par $\mathcal{R} \backslash R_2$.
    
\end{frame}

\begin{frame}{Redondances inter-règles}
   
   Algorithme pour supprimer les redondances inter-règles :
    
    \begin{itemize}
        \item On ajoute le corps de $R$ à une base de fait $\mathcal{F}$
        \item On construit la base de connaissance $\mathcal{KB} = \{\mathcal{F}, \mathcal{R} \backslash R \}$
        \item On sature $\mathcal{KB}$ et on obtient $\mathcal{F}^*$ la base de faits saturée
        \item On gèle la frontière de $R$ et on cherche si la tête de $R$ se trouve dans $\mathcal{F}^*$.
    \end{itemize}
    
    La saturation d'une base de faits étant infinie, il est préférable de prévoir un arrêt au bout d'un certains nombre d'étapes.
    
\end{frame}

\begin{comment}


\begin{frame}{Redondances inter-règles}
   
   Algorithme pour supprimer les redondances inter-règles :
    
    \begin{itemize}
        \color{red}
        \item On ajoute le corps de $R$ à une base de fait $\mathcal{F}$
        \item On construit la base de connaissance $\mathcal{KB} = \{\mathcal{F}, \mathcal{R} \backslash R \}$
        \color{black}
        \item On sature $\mathcal{KB}$ et on obtient $\mathcal{F}^*$ la base de faits saturée
        \item On gèle la frontière de $R$ et on cherche si la tête de $R$ se trouve dans $\mathcal{F}^*$
    \end{itemize}
    
    \begin{block}{Exemple}
          Soit la base de règles $\mathcal{R} = \{ R_1 = p(X) \xrightarrow{} q(X), R_2 = p(X) \xrightarrow{} t(X), R_3 = q(x) \xrightarrow{} t(X)\}$. On commence par traiter $R_1$ et on construit la base de connaissances $\mathcal{KB} = \{ \{p(X)\}, \mathcal{R} \backslash R_1\}$.
    \end{block}
    
\end{frame}

\begin{frame}{Redondances inter-règles}
   
   Algorithme pour supprimer les redondances inter-règles :
    
    \begin{itemize}
        \item On ajoute le corps de $R$ à une base de fait $\mathcal{F}$
        \item On construit la base de connaissance $\mathcal{KB} = \{\mathcal{F}, \mathcal{R} \backslash R \}$
        \color{red}
        \item On sature $\mathcal{KB}$ et on obtient $\mathcal{F}^*$ la base de faits saturée
        \item On gèle la frontière de $R$ et on cherche si la tête de $R$ se trouve dans $\mathcal{F}^*$.
        \color{black}
    \end{itemize}
    
    \begin{block}{Exemple}
          Soit la base de règles $\mathcal{R} \backslash R_1 = \{R_2 = p(X) \xrightarrow{} t(X), R_3 = q(x) \xrightarrow{} t(X)\}$. On sature $\mathcal{F} = \{p(X)\}$ et on obtient $\mathcal{F}^* = \{p(X), t(X)\}$. Ici $q(X)$ n'appartient pas à $\mathcal{F}^*$ donc $R_1$ n'est pas redondante.
    \end{block}
    
\end{frame}

\end{comment}


\begin{frame}{Redondances inter-règles}
   
   Algorithme pour supprimer les redondances inter-règles :
    
    \begin{itemize}
        \color{red}
        \item On ajoute le corps de $R$ à une base de fait $\mathcal{F}$
        \item On construit la base de connaissance $\mathcal{KB} = \{\mathcal{F}, \mathcal{R} \backslash R \}$
        \color{black}
        \item On sature $\mathcal{KB}$ et on obtient $\mathcal{F}^*$ la base de faits saturée
        \item On gèle la frontière de $R$ et on cherche si la tête de $R$ se trouve dans $\mathcal{F}^*$.
    \end{itemize}
    
    \begin{block}{Exemple}
          Soit la base de règles $\mathcal{R} = \{ R_1 = p(X) \xrightarrow{} q(X), R_2 = p(X) \xrightarrow{} t(X), R_3 = q(x) \xrightarrow{} t(X)\}$. On traite $R_2$ et on construit la base de connaissances $\mathcal{KB} = \{ \{p(X)\}, \mathcal{R} \backslash R_2\}$.
    \end{block}
    
\end{frame}

\begin{frame}{Redondances inter-règles}
   
  Algorithme pour supprimer les redondances inter-règles :
    
    \begin{itemize}
        \item On ajoute le corps de $R$ à une base de fait $\mathcal{F}$
        \item On construit la base de connaissance $\mathcal{KB} = \{\mathcal{F}, \mathcal{R} \backslash R \}$
        \color{red}
        \item On sature $\mathcal{KB}$ et on obtient $\mathcal{F}^*$ la base de faits saturée
        \item On gèle la frontière de $R$ et on cherche si la tête de $R$ se trouve dans $\mathcal{F}^*$.
        \color{black}
    \end{itemize}
    
    \begin{block}{Exemple}
          Soit la base de règles $\mathcal{R} \backslash R_2 = \{ R_1 = p(X) \xrightarrow{} q(X), R_3 = q(x) \xrightarrow{} t(X)\}$. On sature $\mathcal{F} = \{p(X)\}$ et on obtient $\mathcal{F}^* = \{p(X), q(X), t(X)\}$. Ici $t(X)$ appartient à $\mathcal{F}^*$ donc $R_2$ est redondante. On peut donc la supprimer de la base de règles.
    \end{block}
    
\end{frame}


\begin{frame}{Redondances inter-règles}
   
  Algorithme pour supprimer les redondances inter-règles :
    
    \begin{itemize}
        \item On ajoute le corps de $R$ à une base de fait $\mathcal{F}$
        \item On construit la base de connaissance $\mathcal{KB} = \{\mathcal{F}, \mathcal{R} \backslash R \}$
     
        \item On sature $\mathcal{KB}$ et on obtient $\mathcal{F}^*$ la base de faits saturée
        \item On gèle la frontière de $R$ et on cherche si la tête de $R$ se trouve dans $\mathcal{F}^*$.
   
    \end{itemize}
    
    \begin{block}{Exemple}
    $R_1$ et $R_3$ n'est pas redondante. 
        Soit la nouvelle base de règles $\mathcal{R}' = \{ R_1 = p(X) \xrightarrow{} q(X), R_3 = q(x) \xrightarrow{} t(X)\}$ sans redondance inter-règles.
    \end{block}
    
\end{frame}


\begin{comment}
\begin{frame}{Redondances inter-règles}
   
   Algorithme pour supprimer les redondances inter-règles :
    
    \begin{itemize}
        \color{red}
        \item On ajoute le corps de $R$ à une base de fait $\mathcal{F}$
        \item On construit la base de connaissance $\mathcal{KB} = \{\mathcal{F}, \mathcal{R} \backslash R \}$
        \color{black}
        \item On sature $\mathcal{KB}$ et on obtient $\mathcal{F}^*$ la base de faits saturée
        \item On gèle la frontière de $R$ et on cherche si la tête de $R$ se trouve dans $\mathcal{F}^*$.
    \end{itemize}
    
    \begin{block}{Exemple}
          Soit la base de règles $\mathcal{R}' = \{ R_1 = p(X) \xrightarrow{} q(X), R_3 = q(x) \xrightarrow{} t(X)\}$. On traite $R_3$ et on construit la base de connaissances $\mathcal{KB} = \{ \{q(X)\}, \mathcal{R} \backslash R_3\}$.
    \end{block}
    
\end{frame}

\begin{frame}{Redondances inter-règles}
   
   Algorithme pour supprimer les redondances inter-règles :
    
    \begin{itemize}
        \item On ajoute le corps de $R$ à une base de fait $\mathcal{F}$
        \item On construit la base de connaissance $\mathcal{KB} = \{\mathcal{F}, \mathcal{R} \backslash R \}$
        \color{red}
        \item On sature $\mathcal{KB}$ et on obtient $\mathcal{F}^*$ la base de faits saturée
        \item On gèle la frontière de $R$ et on cherche si la tête de $R$ se trouve dans $\mathcal{F}^*$.
        \color{black}
    \end{itemize}
    
    \begin{block}{Exemple}
          Soit la base de règles $\mathcal{R}' \backslash R_3 = \{ R_1 = p(X) \xrightarrow{} q(X)$. On sature $\mathcal{F} = \{q(X)\}$ et on obtient $\mathcal{F}^* = \mathcal{F}$. Ici $t(X)$ n'appartient pas à $\mathcal{F}^*$ donc $R_3$ n'est pas redondante. 
    \end{block}
    
\end{frame}




\begin{frame}{Redondances inter-règles}
   
   Algorithme pour supprimer les redondances inter-règles :
    
    \begin{itemize}
        \color{red}
        \item On ajoute le corps de $R$ à une base de fait $\mathcal{F}$
        \item On construit la base de connaissance $\mathcal{KB} = \{\mathcal{F}, \mathcal{R} \backslash R \}$
        \color{black}
        \item On sature $\mathcal{KB}$ et on obtient $\mathcal{F}^*$ la base de faits saturée
        \item On gèle la frontière de $R$ et on cherche si la tête de $R$ se trouve dans $\mathcal{F}^*$
    \end{itemize}
    
    \begin{block}{Exemple}
         La base de règles $\mathcal{R} = \{ R_1 = p(X) \xrightarrow{} q(X), R_2 = p(X) \xrightarrow{} t(X), R_3 = q(x) \xrightarrow{} t(X)\}$ devient $\mathcal{R}' = \{ R_1 = p(X) \xrightarrow{} q(X), R_3 = q(x) \xrightarrow{} t(X)\}$ sans redondances inter-règles.
    \end{block}
    
\end{frame}

\end{comment}